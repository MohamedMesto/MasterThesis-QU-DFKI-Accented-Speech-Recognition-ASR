\section{Introduction}

\acrfull{si} consists on assigning the appropriate subjects to each document of a collection. The subjects describe the content of the document, and can be used to relate documents to one another. Large collections of documents are often hard to navigate, which is why metadata, including subjects, is important. They make the information in the collection more accessible.

Institutional repositories are databases that public institutions, such as universities, use to store and index the outputs of their research \cite{barton2004creating}. Such repositories profit from having a thorough set of subjects that provides structure to its content. Therefore, in this thesis we explore the problem of extending institutional repositories with \acrshort{si}. We focus on repositories that are small and multidisciplinary. By small, we don't refer to hundreds of documents, but rather tens of thousands, which is considered small in the era of big data and deep learning.

We have constructed a dataset for our experiments with the repositories of Berlin's three universities: the \acrfull{tu}, the \acrfull{hu} and the \acrfull{fu}. They all have institutional repositories, called depositonce, edoc and refubium, respectively. The medical university, called Charité, also uses refubium to post their work online. All three repositories offer their data through the Open Archives Initiative Protocol for Metadata Harvesting (OAI-PMH)\footnote{\url{https://www.openarchives.org/pmh/}}.

These repositories already have subjects, but they are of little use. Over 80 \% of them appear only once and therefore cannot be used to relate documents. Furthermore, the subjects are different in each repository. This hinders their interoperability, which is one of the goals of public repositories \cite{barton2004creating}. This thesis aims to increase the quality of the \acrshort{si} and to standardize it in such a way that documents from different repositories can be related to one another.

The easiest way of introducing high quality subjects in different repositories is using an external source. We retrieve 2,157 subjects from the \acrfull{mag} \cite{shen2018web}, evenly spread across its 19 fields of study. This subset covers broad topics of numerous disciplines, such as \textit{Physics} and \textit{Philosophy}, which makes it ideal for our multidisciplinary dataset. We focus specifically on machine learning approaches, implementing methods of the two larger paradigms: supervised and unsupervised approaches.

Our unsupervised approach closely follows the \acrshort{si} procedure of \acrshort{mag}. The \acrshort{mag} also handles multidisciplinary data and is, to the best of our knowledge, the most successful unsupervised method for \acrshort{si}, with an accuracy of 80 \%. Documents are enriched with data from their venues, advisors and referees, whereas subjects are represented with text from their corresponding Wikipedia articles. Then, all these representations are vectorized using word embeddings trained with the skip-gram model \cite{mikolov2013distributed}. Subjects are finally assigned to documents by evaluating their similarity in vector space.

Our supervised approach consists of a convolutional neural network, which we train with over 200,000 scientific texts from \acrshort{mag} that are assigned the subjects of our subset. The model learns to assign subjects to documents on a larger dataset, and is then applied to the documents from the repositories. Our model comes from a similar use case \cite{gargiulo2019deep}, where the subjects also form a hierarchy and are assigned to scientific documents.

We extend the model in two different ways. We first address the asymmetry that arises from documents being assigned only a few from many subjects, and then by ensuring that the outputs of the model don't violate the subject hierarchy. The three resulting models have only minimal differences, but they still yield different results in our evaluation sets. The original model performed best.

The best supervised model is then compared with the unsupervised approach on three evaluation sets that we have extracted from the repositories. Two of these sets consider only the 19 fields of study, covering more than 70 \% of all documents. The other set considers all 2,157 subjects. It covers 24 \% of the documents, with mostly one subject assigned to each document.

There are numerous metrics used in \acrshort{si}. Many of them can be adapted to consider the subject hierarchy, which introduces relationships among the concepts. The evaluation of a subject-document pair is not binary anymore, as certain pairs are better than others when considering the distance of the guess to the correct subjects in the hierarchy. Our evaluation set differs from the common scenario in its sparsity: documents appear mostly once in each evaluation set, meaning that we don't have the full set of subjects for the documents. We consider this when designing our evaluation metrics, which comprise a flat metric and a hierarchical one.

Our experiments show that the supervised method is significantly more accurate. We therefore conclude that supervised approaches are better suited for \acrshort{si} tasks in small and multidisciplinary repositories. They also have the most potential and, considering our two approaches, they are easier to implement.

\subsection{Outline}

The following chapters are structured as followed. Chapter \ref{problem} starts with the introduction of the problem tackled by this thesis. We then present its scope, defining which types of documents we index, and which \acrshort{si} methods we consider. We also present the 2,157 \acrshort{mag} subjects that we will use to index the documents. Finally, we present our objectives by formulating a research question.

We then perform an analysis of the institutional repositories of the three Berlin Universities in chapter \ref{repo_analysis}. We evaluate the existing subjects of the repositories, concluding that most of them cannot be used to relate documents to one another. We explore venues, advisors and referees, as these will be used to relate documents in the unsupervised approach. We finally assess the quality of the data, finding that 6 \% of the documents don't have an abstract and many others have abstracts in languages other than English.

Chapters \ref{subject_indexing} and \ref{hmc} present the literature on unsupervised and supervised approaches, respectively. Chapter \ref{subject_indexing} also presents several ways of classifying \acrshort{si} methods. They mainly consider the existence of a set of subjects and the availability of training data. Chapter \ref{hmc} presents the work on \acrfull{hmc}, where objects assigned a variable number of labels, and the labels are structured as a hierarchy. Objects may be images, proteins or, as in our case, text. \acrshort{si} is an application of this problem. This chapter presents the common ways of tackling this classification problem, as well as several implementations.

Our unsupervised approach is presented in chapter \ref{unsupervised_approach}, whereas the supervised approach is outlined in chapter \ref{supervised_approach}. Both were already introduced above. The supervised and unsupervised approaches are compared in chapter \ref{eval}, where our three evaluation sets and two metrics are also presented. We end this thesis with chapter \ref{conclusion}, where we discuss the differences between the approaches regarding their performance, implementation and interpretability. We also define the applicability of the supervised approach, which performs best.