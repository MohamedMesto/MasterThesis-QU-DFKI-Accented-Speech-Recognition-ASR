\subsection{Introduction} \label{problem_intro}

As discussed in chapter \ref{repo_analysis}, the subjects that are currently present in the repositories are not well maintained. 81 \% of all subjects occur only once, i.e. they are only used by one publication. These cannot be used to relate documents. Furthermore, only 5 \% of the subjects are used by more than 3 publications. The poor quality of the subjects decreases the accessibility of the repositories. Each of them contains thousands of documents, and navigating them can be a cumbersome task. Subjects can be a useful tool for finding the desired documents, and also for looking for documents that are related to the ones we are currently reading.

Consider a repository where there is a maintained set of subjects. Each document in the repository is assigned a subset of these subjects, which can be navigated through the document's page in the repository. If a user is currently reading a publication about brain-computer interfacing, and would like to read more about that topic, the user could browse other documents that handle it by clicking on the corresponding subject. Furthermore, if the user has trouble understanding one of the building blocks of the documents, the user can click on the corresponding subject to search for other documents that maybe present that concept in more detail.