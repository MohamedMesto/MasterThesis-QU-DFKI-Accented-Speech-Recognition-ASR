\subsubsection{Subject Indexing} \label{problem_scope_asi}

In chapter \ref{subject_indexing}, we discuss the different ways in which \acrshort{si} approaches can be classified. For now, we only require the distinction between \textit{assigned SI} and \textit{derived SI}, which differ on the subjects that are used to index the documents. In assigned \acrshort{si}, an existing set of subjects is used. The set can be constructed manually or retrieved from another application. On the other hand, derived \acrshort{si} extracts the subjects from the documents using only intrinsic information, such as term frequency.

Assigned \acrshort{si} is the preferred approach for information retrieval because it offers better precision and recall, given that users don't always know exactly the subjects they are looking for \cite{golub2019automatic}. Furthermore, using the same set of subjects in all three repositories allows them to be searched together. If each repository had its own set of subjects, their accessibility would be hindered. This is why standard sets of subjects, such as the \acrfull{ddc}, are so useful.

Given the heterogeneity and the technical nature of our dataset, derived \acrshort{si} approaches would not reach the desired quality. Our heterogeneous dataset limits the performance of term-weighting methods, such as TF-IDF, as phrases that are relevant because of their scientific meaning may not appear often enough to be noticed by these methods. Because of this, and given the availability of a suitable set of subjects, we will focus solely on methods that perform assigned \acrshort{si}. The existing subjects in the repositories will be used only for evaluation purposes.

We will use the subjects from \acrfull{mag}, a scientific knowledge graph presented in section \ref{subject_indexing_mag}. The set Microsoft has built covers all the disciplines present in the repositories and is therefore well suited for our task. It also includes a hierarchy, which we will use when training classification models. In the following section, we further discuss the \acrshort{mag} set of subjects. We argue why a subset of them suffices for our approach and which size and composition we require. We then present the final set of subject we use in this thesis.