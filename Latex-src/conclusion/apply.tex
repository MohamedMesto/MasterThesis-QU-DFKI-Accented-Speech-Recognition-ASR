\subsection{Applicability}

We now consider the applicability of the supervised model, which is our choice for performing \acrshort{si} in small and multidisciplinary repositories. Our model can be immediately used in any repository, without any further modification, given the size of its vocabulary and the amount of disciplines it covers. The model can also be modified and trained again if some parameters require tuning. For instance, some repositories may benefit from using more words of each text, instead of the 250 that our model uses. The most costly aspect of training the models again is processing and vectorizing the documents of the \acrshort{mag}, which we have already done. Training the model takes around 10 hours with our resources.

Given the current hit rate of the model, we don't recommend using it to automatically assign subjects to documents, as there would be numerous erroneous subject assignments. Rather, the model can be used to assist users when looking for the appropriate subjects for their document, as proposed in \cite{heryawan2021medical}. This is useful in scenarios like ours, where the large number of subjects makes it hard to pick them from a list. Once the hit rate improves, the model could be used for automatic indexing. The assignment probabilities can be used as a filter, where the user decides the lower threshold for a search. The subjects of each document can also be displayed in their respective pages in the repositories, along with their corresponding probabilities.