\subsubsection{Golub's classification} \label{subject_indexing_golub}

Golub classifies \acrshort{si} approaches regarding the application purpose, the amount of research backing it and the paradigm used (\acrlong{ml} or string matching, essentially) \cite{golub2019automatic}. It classifies \acrshort{si} approaches into three groups.

\textit{Document clustering} is suited for cases where there is neither training data nor a \acrshort{kos}. Clustering algorithms group documents according to a given similarity metric. They can be used to group documents that belong to the same topic. For example, the similarity metric may measure the amount of mutual information between documents \cite{slonim2002unsupervised}. Document clustering poses two challenges: the resulting structures may be hard to understand \cite{chen2000bringing}, and they can also change once more documents are added. The second challenge is that the assignment of topics to clusters may be unclear and require human intervention.

\textit{Text categorization} is used when both training data and a \acrfull{kos} are available. It consists of a supervised \acrshort{ml} algorithm that learns the features of the subjects of the \acrshort{kos} from the already assigned documents. Once the algorithm is trained, these features will enable it to map subjects to new documents. It can be applied for \acrshort{kos}s that order subjects in hierarchies. Furthermore, doing so may improve the classification accuracy of the method \cite{chen2000bringing}. The third group, \textit{document classification}, is similar to text categorization, but differs from it in the quality of the \acrshort{kos}. The higher \acrshort{kos} quality enables document classification methods to rely on string matching to map subjects to documents, instead of complex machine learning algorithms \cite{khoo2015augmenting}.
