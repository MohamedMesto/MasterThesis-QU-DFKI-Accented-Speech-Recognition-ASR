\subsubsection{Updates since the paper's publication} \label{mag_updates}

Last year, the research group posted a blog post\footnote{\url{https://www.microsoft.com/en-us/research/project/academic/articles/expanding-concept-understanding-in-microsoft-academic-graph/}} highlighting some changes to the methods described above. There were two major updates to the phase of discovering concepts, which we discuss in this section. These two updates increased the number of concepts from 227 thousand to over 700 thousand. The second update is also performed regularly to discover new topics.

The first update consisted of adding concepts from the biomedical domain using the  Unified Medical Language System (UMLS) vocabulary. Biomedical concepts from the (UMLS) that were relevant to the corpus (regarding term frequency) were added to the concept model. This procedure was performed only once. The second update is the paradigm shift for topic discovery. Instead of retrieving concepts from Wikipedia, the new approach directly extracts concepts from the academic literature. It consists of two steps.

\begin{enumerate}
    \item Identify words or phrases in the text of the publications that are concepts (without stating which specifically). This problem is called \textit{sequence labeling}. The authors' approach consists of doing lexical matching on a sample of \acrshort{mag} publications using the synonyms of existing concepts to generate a training set. This set is then fed to a transformer as a context encoder and a conditional random field as a tag decoder to train a binary classifier on each word.
    \item Differentiate between existing, new and low-quality concepts by evaluating the relevance of the URLs returned by the Bing Web Search \acrshort{api} when looking for the concept. Equivalent concepts are grouped together.
\end{enumerate}