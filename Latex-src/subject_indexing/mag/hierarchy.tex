\subsubsection{Hierarchy building} \label{mag_hierarchy_building}

Once the publications have been tagged with the concepts, they can be used to extract a hierarchy for the concepts. The authors use \textit{subsumption} for this purpose, which assumes that concept $x$ subsumes concept $y$ if $y$ only occurs in a subset of the publications where $x$ occurs \cite{sanderson1999deriving}. The condition can be relaxed; $y$ is considered a child of $x$ in the hierarchy if both terms co-occur in a certain proportion of publications, i.e. 75 \%. The authors extend this method by including the confidence scores for the pairs of concepts and publications computed in the previous phase.

The resulting subsumption model was used to construct a six-level hierarchy of concepts. The relationships between the concepts of the first two levels were adjusted manually. All other relationships were left as they were output by the model. A limitation of this model, which translates to a worse accuracy (see table \ref{tab:mag}), is that relationships are not always transitive. For example, Fernando Alonso is a Formula 1 driver, which is a profession, but the relationship between ``Fernando Alonso'' and ``profession'' is unclear. The authors stated that they plan on addressing this issue by including information about the types of the entities.