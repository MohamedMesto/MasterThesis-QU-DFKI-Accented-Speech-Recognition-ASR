\subsubsection{String matching}

String matching methods rely on the quality and thoroughness of the \acrfull{kos} they use \cite{golub2019automatic}. A \acrshort{kos} is a special set of subjects where the subjects are structured, either in a hierarchy or through other kinds of relationships. The \acrlong{ddc} is an example of a  qualitative \acrshort{kos}. When the \acrshort{kos} covers all the relevant keywords that indicate that a document handles a certain topic, string matching methods may suffice to map subjects to documents. They are easier to implement and interpret, as well as cheaper to compute than more complex mapping functions, such as those optimized by machine learning algorithms.

An early application of this approach is the Wordsmith project \cite{godby2001wordsmith}. Its objective was to extract noun phrases from documents, which were then compared with the \acrfull{ddc} \acrshort{kos}. The authors thought that removing all other words during pre-processing would improve the accuracy of the indexing. Unfortunately, no significant difference in performance was observed.

Another project assigned \acrshort{ddc} terms to documents in unrelated libraries to facilitate cross-search \cite{khoo2015augmenting}. They achieved the best indexing results when using the title, the description and the keywords to represent each document. Their simple string matching approach took advantage of \acrshort{ddc} hierarchies for disambiguation to achieve competitive results, even when compared with \acrshort{ml} approaches that require annotated documents.