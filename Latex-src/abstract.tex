\begin{abstract}
This thesis evaluates the performance of supervised and unsupervised methods for performing subject indexing on institutional repositories, which are small and multidisciplinary. Our unsupervised approach is inspired by the \acrfull{mag}, where the documents and subjects are vectorized and compared in vector space. For the supervised approach, we implement a convolutional neural network and train it on a set of \acrshort{mag}'s indexed documents. The two methods are evaluated with data extracted from the repositories of Berlin's three universities, which show that the supervised approach is more accurate. Its design enables it to be used in other repositories without any modifications.
\end{abstract}

\vspace{1cm}

\selectlanguage{ngerman}
\begin{abstract}
In dieser Arbeit wird die Leistung von überwachten und unüberwachten Methoden für die thematische Indexierung von kleinen und multidisziplinären institutionellen Repositorien bewertet. Unser unüberwachter Ansatz orientiert sich am \acrfull{mag}, bei dem die Dokumente und Themen vektorisiert und im Vektorraum verglichen werden. Für den überwachten Ansatz implementieren wir ein faltendes neuronales Netzwerk und trainieren es auf einer Reihe von indexierten Dokumenten des \acrshort{mag}. Die beiden Methoden werden mit Daten aus den Repositorien der drei Berliner Universitäten evaluiert. Daraus zeigt sich, dass der überwachte Ansatz genauer ist. Sein Design ermöglicht es, ihn ohne Änderungen in anderen Repositorien zu verwenden.
\end{abstract}
\selectlanguage{english}